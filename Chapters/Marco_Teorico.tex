\chapter{Marco Teórico}
\begin{itemize}
    \item[2.1.] Inteligencia Artificial
    \begin{itemize}
        \item[2.1.1.] Concepto de Inteligencia Artificial
        \item[2.1.2.] Machine Learning
    \end{itemize}
    \item[2.2.] Deep Learning
    \begin{itemize}
        \item[2.2.1.] Fundamentos
        \item[2.2.2.] Redes Neuronales Convolucionales (CNN)
        \item[2.2.3.] Redes Neuronales Recurrentes (RNN) 
        \item[2.2.4.] Long Short-Term Memory (LSTM)
        \item[2.2.5.] Transformers
    \end{itemize}
    \item[2.3.] Métodos de evaluación de modelos de Machine Learning
    \item[2.4.] Accidentes de tránsito
    \begin{itemize}
        \item[2.4.1.] Definición y clasificación según la Ley Nacional de Tránsito
        \item[2.4.2.] Estadísticas de Accidentes de Tránsito en Argentina
        \item[2.4.3.] Predicción del riesgo de accidentes de tránsito
        \item[2.4.4.] Trabajos relacionados
    \end{itemize}
    \item[2.5.] Visualización de Datos Geoespaciales
        \begin{itemize}
            \item[2.5.1.] Importancia de la Visualización en el Análisis de Datos
            \item[2.5.2.] Tecnologías y Herramientas de Visualización
        \end{itemize}
    \item[2.6.] Metodologías Ágiles para el Desarrollo de Software
    \begin{itemize}
        \item[2.6.1.] Principios de las Metodologías Ágiles
        \item[2.6.2.] Kanban / Personal Kanban
    \end{itemize}
\end{itemize}